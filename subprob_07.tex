\documentclass[12pt, a4paper]{article}
\usepackage[utf8]{inputenc}
\usepackage[brazilian]{babel} % Hifenização e dicionário
\usepackage[left=3.00cm, right=2.00cm, top=3.00cm, bottom=2.00cm]{geometry}
\usepackage{enumitem} % Para itemsep etc
\usepackage{longtable} % Dependência do longtabu
\usepackage{tabu} % Para melhor criação de tabelas
\usepackage{listings} % Para códigos
\usepackage{lstautogobble} % Códigos indentados corretamente
\usepackage{color} % Para coloração de códigos
\usepackage{zi4} % Para fonte de códigos
\usepackage{parskip} % Linha em branco entre parágrafos em vez de recuo
\usepackage{graphicx}
\usepackage{verbatim}
\usepackage[breaklinks]{hyperref}

\setlist[enumerate]{leftmargin=*}

% Um dia, talvez: http://linorg.usp.br/CTAN/macros/latex/contrib/minted/minted.pdf

\tabulinesep=0.75ex % Espaçamento interno das tabelas

\definecolor{bluekeywords}{rgb}{0.13, 0.13, 1}
\definecolor{greencomments}{rgb}{0, 0.5, 0}
\definecolor{redstrings}{rgb}{0.9, 0, 0}
\definecolor{graynumbers}{rgb}{0.5, 0.5, 0.5}

\usepackage{listings}
\lstset{
    autogobble,
    columns=fullflexible,
    showspaces=false,
    showtabs=false,
    breaklines=true,
    showstringspaces=false,
    breakatwhitespace=true,
    escapeinside={(*@}{@*)},
    commentstyle=\color{greencomments},
    keywordstyle=\color{bluekeywords},
    stringstyle=\color{redstrings},
    numberstyle=\color{graynumbers},
    basicstyle=\ttfamily\footnotesize,
    frame=l,
    framesep=12pt,
    xleftmargin=12pt,
    tabsize=4,
    captionpos=b
}

\begin{document}
\begin{center}
    \textsc{Universidade Federal do Rio Grande do Norte} \\
    \textsc{Departamento de Informática e Matemática Aplicada}
\end{center}

\bigskip

\begin{tabular}{@{}ll@{}}
    \emph{Disciplina:} & DIM0437 --- Linguagens de Programação:
                            Conceitos e Paradigmas \\
    \emph{Docente:}    & Umberto Souza da Costa \\
    \emph{Discentes:}  & Dogival Ferreira da Silva Junior \\
                       & Felipe Cortez de Sá \\
                       & Gabriel Sebastian von Conta \\
                       & Phellipe Albert Volkmer \\
                       & Vinícius Araújo Petch
\end{tabular}

\bigskip

\begin{center}
\large \textbf{Subproblema 7} --- Subprogramas
\end{center}

\section{Problema}
    \subsection{Produto do problema}
    Definição da sintaxe e semântica intuitiva dos mecanismos que regem as
    formas de abstração de processamento (procedimentos e funções) da
    linguagem de programação a ser definida pelo grupo. Incluir a
    representação destas abstrações, assim como os mecanismos de passagem
    de parâmetros e a implementação sugerida.

    \subsection{Questões}
    \begin{enumerate}
        \item
        Como serão definidos os procedimentos e funções de sua linguagem?
        Note que estes conceitos são diferentes, embora sejam tratados de
        forma unificada por algumas linguagens de programação. Quais são as
        diferenças entre esses conceitos?

        \item
        Como serão definidos os parâmetros da linguagem? Quais dados
        poderão ser colocados como argumentos em chamadas a procedimentos e
        funções? Nomes de subprogramas poderão ser utilizados como
        parâmetros?

        \item
        Quais serão as formas de passagem de parâmetros e como serão
        implementadas?

        \item
        Sua linguagem verificará os tipos de parâmetros dos subprogramas?

        \item
        A linguagem terá subprogramas sobrecarregados ou genéricos?

        \item
        A linguagem deverá ter compilação separada ou independente?

        \item
        Sua linguagem dará suporte a co-rotinas?
    \end{enumerate}

\section{Resoluções}
    \begin{enumerate}
        \item
        Na nossa linguagem, daremos suporte tanto a funções quanto
        procedimentos, procurando explicitar sua diferença através da sintaxe
        com o objetivo de deixar claros ambos os conceitos facilitando o
        aprendizado dos nossos usuários.

        Tanto funções quanto procedimentos são sub-rotinas, ou seja, sequências
        de comandos reutilizáveis que podem ser alteradas sem a necessidade de
        sua repetição no código-fonte. A diferença conceitual entre
        procedimento e função reside na presença de variáveis de retorno para
        funções. Os procedimentos, por outro lado, são úteis apenas quando
        geram efeitos colaterais.

        Em C, por exemplo, procedimentos possuem a mesma sintaxe de funções,
        com \texttt{void} no lugar do tipo:

        \begin{lstlisting}[language=C]
            int funcao(int a, int b) {
                return a + b;
            }

            void procedimento() {
                puts("Este comando gera um efeito colateral");
            }
        \end{lstlisting}

        A sintaxe da nossa linguagem, portanto, estará na seguinte forma

        \begin{lstlisting}[language=C]
            <function>  ::= function <id> "(" <parameters_list> ")" "{" <stmts> <return_stmt> "}"

            <procedure> ::= procedure <id> "(" <parameters_list> ")" "{" <stmts> "}"
        \end{lstlisting}

        \item
        Os parâmetros em nossa linguagem poderão ser de qualquer tipo da
        linguagem, inclusive os tipos criados pelo usuário. A utilização de
        parâmetros será de forma posicional, uma vez que, como a maior parte
        dos programas será curto, não são necessários keyword parameters, que
        diminuiriam a capacidade de escrita do nosso código e o tornaria muito
        extenso. Subprogramas não poderão ser utilizados como argumentos, uma
        vez que acarretaria em uma queda de legibilidade para o programa. Além
        disso, novamente, como o caráter dos subprogramas é simples, pode-se
        facilmente atribuir o valor de um subprograma a uma variável e então
        usar essa variável como argumento.

        \item
        Como decidimos usar variáveis de referência em nossa linguagem, teremos
        necessariamente passagem por referência, ou seja, o endereço do
        argumento é passado para o subprograma e colocado na stack. Também
        teremos passagem por valor com cópia, em que uma cópia do argumento é
        criada na stack ao ser chamado o subprograma.

        \item
        Sim, ela utilizará o método de protótipo, onde os tipos dos parâmetros são inclusos dentro da lista de parâmetros, e por meio deste o tipo de um parâmetro é checado. Em C99 e C++, é possível evitar a checagem de tipo para alguns parâmetros, porém essa funcionalidade não será utilizada.
        Exemplo de prototipagem:
        \begin{lstlisting}[language=C]
            int function1 (int par1, int par2, float par3) {
            	...
            }
        \end{lstlisting}
        Neste exemplo ocorre evasão de checagem de tipo, onde \emph{printf} checará apenas o primeiro parâmetro. Após isso, tudo é permitido (inclusive nada):
        \begin{lstlisting}[language=C]
            int printf(const char* format_string, . . .);
        \end{lstlisting}
        
        \item
        Subprograma genérico é um tipo de subprograma que funciona para
        diferentes tipos de entrada em diferentes ativações, é implementado com
        funções de template,  definindo um grupo de funções que pode ser
        gerado. Cabe ao compilador decidir de acordo com os parâmetros no
        momento da chamada da função qual exemplar do grupo será usado.

        Subprogramas por sobrecarga, por outro lado, baseiam-se em um grupo de
        subprogramas de mesmo identificador declarado várias vezes no esqueleto
        do programa. Uma função específica do grupo é usada dependendo dos
        parâmetros quando a função é chamada.

        Na nossa linguagem optamos por usar somente de sobrecarga, ignorando
        subprogramas genéricos. Essa decisão advém da maior simplicidade em se
        definir sobrecarga, mais adequado aos usuários de nossa linguagem.

        \begin{comment}
            Com sobrecarga:
            int soma(int a, int b) {
                return a+b;
            }

            float soma(float a, float b) {
                return a+b;
            }

            int soma(String a, String b) {
                return a^b;
            }
        \end{comment}

        Subprogramas por sobrecarga, por outro lado, baseiam-se em um grupo de
        subprogramas de mesmo identificador declarado várias vezes no esqueleto
        do programa. Uma função específica do grupo é usada dependendo dos
        parâmetros quando a função é chamada.

        Na nossa linguagem optamos por usar de sobrecarga somente, ignorando
        subprogramas genéricos. Essa decisão advém da maior simplicidade em se
        definir sobrecarga, mais adequado aos usuários de nossa linguagem.

        \item
        \begin{comment}
            (Compilação separada: as unidades de compilação podem ser
            compiladas em tempos diferentes, mas elas não são independentes uma
            da outra se qualquer uma delas acessar ou usar quaisquer entidades
            da outra. Tal interdependência é necessária se precisar ser feita
            verificação de interface.) (Compilação independente: unidades de
            programa podem ser compiladas sem informações sobre quaisquer
            outras unidades de programa.  Unidades compiladas separadamente não
            são verificadas quanto à coerência de tipos) (Algumas linguagens
            não oferecem nem compilação separada, nem compilação independente,
            significando que somente a unidade de compilação é um programa
            completo. Isso a torna virtualmente inútil para aplicações
            industriais)
        \end{comment}

        Na compilação separada, partes do código podem ser compiladas em tempos
        diferentes, desde que essas partes não possuam outras dependências
        externas. Já na compilação independente, essa dependência não importa:
        qualquer unidade de código pode ser compilada sem se preocupar com
        dependências. Porém, isso faz com que unidades compiladas separadamente
        não tenham verificação quanto à coerência de tipo. Caso a linguagem não
        possua nenhum dos dois tipos de compilação citados, ou seja, compilação
        única, ela se tornará virtualmente inútil para aplicações industriais.

        Nossa linguagem terá compilação separada, pois essa compilação é
        bastante prática já que caso ocorra uma alteração no código, nem sempre
        será necessário compilar todo o código. A escolha da compilação
        separada ao invés da compilação independente se dá na falta de
        verificação de coerência de tipo desta.

        \item
        Uma co-rotina é um tipo especial de subprograma. Ao invés de possuir
        uma relação mestre-escravo entre o subprograma que chama e o
        subprograma chamada, ambos estão em uma relação mais justa.

        Apesar de co-rotinas serem uma funcionalidade relevante a uma
        linguagem, como a nossa linguagem possui um escopo educativo elas não
        serão utilizadas, e portanto desnecessárias.

    \end{enumerate}
\end{document}

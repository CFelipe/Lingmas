\documentclass[12pt, a4paper]{article}
\usepackage[utf8]{inputenc}
\usepackage[brazilian]{babel} % Hifenização e dicionário
\usepackage[left=3.00cm, right=2.00cm, top=3.00cm, bottom=2.00cm]{geometry}
\usepackage{enumitem}
\usepackage{longtable} % Dependência do longtabu
\usepackage{tabu} % Para melhor criação de tabelas
\usepackage{color} % Para coloração de códigos
\usepackage{parskip} % Linha em branco entre parágrafos em vez de recuo
\usepackage[breaklinks]{hyperref}

\extrarowsep=2ex % Espaçamento interno das tabelas
\tabulinesep=1ex % Espaçamento interno das tabelas

\begin{document}

\begin{center}
\large \textbf{Subproblema 8} --- Abstrações --- Tabela
\end{center}

\bigskip

\begin{longtabu}{| X |}
    \hline
    \begin{center}
        \large \textbf{Lingmas} \linebreak
        Projeto de uma linguagem de programação
    \end{center}
    \\ \hline

    \begin{tabu}{ X | X }
        \textbf{Dados} &
        \textbf{Noções faltantes}
        \\
        \begin{minipage}[t]{\linewidth}
        \begin{itemize}[itemsep=.5ex,parsep=.0ex,after=\strut,leftmargin=15pt]
            \item Módulos
            \item Pacotes
            \item Encapsulamento de construção
            \item Encapsulamento de nomes
        \end{itemize}
        \end{minipage}
        &
        \begin{minipage}[t]{\linewidth}
        \begin{itemize}[itemsep=.5ex,parsep=.0ex,after=\strut,leftmargin=15pt]
            \item Abstração
            \item Tipos abstratos de dados
            \item Classes
            \item Arquivos de cabeçalho
            \item Namespaces
            \item Encapsulamento
        \end{itemize}
        \end{minipage}
    \end{tabu}
    \\ \hline

    \textbf{Passos para a solução} \newline
    \begin{minipage}[t]{\linewidth}
    \begin{itemize}[itemsep=.5ex,parsep=.0ex,after=\strut,leftmargin=15pt]
        \item
        Leitura do material recomendado pelo professor
        \item
        Discussão em grupo sobre o assunto
        \item
        Uso da internet para uma pesquisa mais extensiva
        \item
        Compartilhamento de links e marcação de comentários oportunos em
        documento compartilhado
        \item
        Resposta das questões propostas pelo professor
        \item
        Atualização do repositório com iterações incrementais de documentos
    \end{itemize}
    \end{minipage}
    \\ \hline

    \textbf{Resumo da solução} \newline
    Foi feita a leitura do livro com discussões em grupo e com o professor.
    Atentamo-nos às questões dos subproblemas e respondendo-as, percebemos como
    se relacionavam com as características da nossa linguagem. Utilizamos a
    planilha criada no subproblema 6 para auxílio das definições de
    características deste subproblema.
    \\ \hline

    \textbf{Conclusões} \newline
    Na realização desse trabalho, estudamos a importância de abstrações para as
    linguagens de programação, a diferença entre módulos, pacotes e classes, os
    conceitos de genericidade, construção de encapsulamento e encapsulamento de
    nomes.
    \\ \hline

\end{longtabu}

\end{document}

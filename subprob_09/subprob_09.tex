\documentclass[12pt, a4paper]{article}
\usepackage[utf8]{inputenc}
\usepackage[brazilian]{babel} % Hifenização e dicionário
\usepackage[left=3.00cm, right=2.00cm, top=3.00cm, bottom=2.00cm]{geometry}
\usepackage{enumitem} % Para itemsep etc
\usepackage{longtable} % Dependência do longtabu
\usepackage{tabu} % Para melhor criação de tabelas
\usepackage{listings} % Para códigos
\usepackage{lstautogobble} % Códigos indentados corretamente
\usepackage{color} % Para coloração de códigos
\usepackage{zi4} % Para fonte de códigos
\usepackage{parskip} % Linha em branco entre parágrafos em vez de recuo
\usepackage{graphicx}
\usepackage{verbatim} % Para comentários
\usepackage[breaklinks]{hyperref}

% Um dia, talvez: http://linorg.usp.br/CTAN/macros/latex/contrib/minted/minted.pdf

\tabulinesep=0.75ex % Espaçamento interno das tabelas

\definecolor{bluekeywords}{rgb}{0.13, 0.13, 1}
\definecolor{greencomments}{rgb}{0, 0.5, 0}
\definecolor{redstrings}{rgb}{0.9, 0, 0}
\definecolor{graynumbers}{rgb}{0.5, 0.5, 0.5}
\definecolor{graybox}{rgb}{0.9, 0.9, 0.9}

\newcommand{\inlinecode}[1]{\textbf{\lstinline{#1}}}

\usepackage{listings}
\lstset{
    autogobble,
    columns=fullflexible,
    showspaces=false,
    showtabs=false,
    breaklines=true,
    showstringspaces=false,
    breakatwhitespace=true,
    escapeinside={(*@}{@*)},
    commentstyle=\color{greencomments},
    keywordstyle=\color{bluekeywords},
    stringstyle=\color{redstrings},
    numberstyle=\color{graynumbers},
    basicstyle=\ttfamily\footnotesize,
    frame=l,
    framesep=12pt,
    xleftmargin=12pt,
    tabsize=4,
    captionpos=b
}

\begin{document}
\begin{center}
    \textsc{Universidade Federal do Rio Grande do Norte} \\
    \textsc{Departamento de Informática e Matemática Aplicada}
\end{center}

\bigskip

\begin{tabular}{@{}ll@{}}
    \emph{Disciplina:} & DIM0437 --- Linguagens de Programação:
                            Conceitos e Paradigmas \\
    \emph{Docente:}    & Umberto Souza da Costa \\
    \emph{Discentes:}  & Dogival Ferreira da Silva Junior \\
                       & Felipe Cortez de Sá \\
                       & Gabriel Sebastian von Conta \\
                       & Phellipe Albert Volkmer \\
                       & Vinícius Araújo Petch
\end{tabular}

\bigskip

\begin{center}
\large \textbf{Subproblema 9} --- Adaptação ao paradigma funcional
\end{center}

\section{Problema}
    \subsection{Produto do problema}
    O produto desta fase deve ser um documento especificando uma versão
    funcional da linguagem originalmente definida pelo grupo. O documento deve
    deixar evidentes as adaptações necessárias para que a linguagem migre de
    paradigma.

    \subsection{Questões}
    \begin{enumerate}[leftmargin=*]
        \item % r_01.tex
        Quais são as características principais do paradigma funcional? Analise
        e compreenda todos os conceitos deste paradigma antes de responder às
        questões seguintes.

        \item % r_02.tex
        Quais diferenças existem entre as noções de variáveis dos paradigmas
        imperativo e funcional? Qual é o impacto dessas diferenças no projeto
        de uma linguagem? Qual é o impacto dessas diferenças na pragmática da
        linguagem?

        \item % r_03.tex
        Quais características da linguagem originalmente projetada poderão
        permanecer quando da mudanção para o paradigma funcional?

        \item % r_04.tex
        Quais características da linguagem originalmente projetada que
        precisarão ser retiradas quando da mudança para o paradigma funcional?

        \item % r_05.tex
        Quais novas características poderão ser integradas à linguagem?

        \item % r_06.tex
        Dê exemplos de programas escritos na versão funcional de sua linguagem.

    \end{enumerate}

\section{Resoluções}
    \begin{enumerate}[leftmargin=*]
        \item
        Uma \textbf{abstração} é uma simplificação útil para o programador, uma vez que
permite representar uma entidade, filtrando as informações e operações,
aproveitando apenas o que é relevante.

Por exemplo, uma entidade \inlinecode{grupo de lpcp} possui os atributos gerais
componentes e nota e trabalhos, mas também possui características únicas a
certos grupos, como por exemplo \inlinecode{possui gordos}. Criamos então um objeto
\inlinecode{G2} que precisa do atributo específico \inlinecode{possui gordos}, mas não
\inlinecode{componentes}, \inlinecode{nota} e \inlinecode{trabalhos}, herdados da entidade
\inlinecode{grupo de lpcp}.


        \item
        \begin{enumerate}
        \item
        No caso de tipos abstratos de dados:

        Um \emph{tipo abstrato de dado} tem a representação do tipo dos objetos
        ocultas para as unidades de programa que utilizam esse tipo. Dessa forma as
        únicas operações possíveis nesses objetos são aquelas que estão na
        definição do objeto.

        Além disso as operações do tipo e as operações definidas nos objetos
        desse tipo, que fornecem a interface desse tipo, são contidas em uma
        única unidade sintática.

        Assim,  a interface do tipo não depende da representação dos objetos
        nem da definição de suas operações. Além disso outras unidades de
        programa podem usar variáveis do tipo definido.

        Na nossa linguagem, utilizaremos tipos abstratos de dados, presentes
        tanto no tipo float, quanto em tipos criados pelo usuário.

        % Abstrai a representação e funções de manipulação dessa representação.
        % Amadurecimento de tipo abstrato de dados -> Classes. Pois permite
        % agrupar abstrações de dados e funções
        \item
        No caso de pacotes:

        % Módulos e pacotes permitem maior organização do código e sua consequente
        % repartição
        \item No caso de pacotes:

        Um \emph{pacote} é como a implementação de módulo é chamada na
        linguagem Ada. Nela, os cabeçalhos e corpos dos módulos podem ser
        separados.

        Nós utilizaremos pacotes (packages). A ideia de pacotes será
        implementada semelhante à utilização de módulos em Ada, onde cabeçalho
        e corpo podem ser separados. Porém, na nossa implementação cabeçalho e
        corpo terão que estar no mesmo arquivo.

        \item
        No caso de classes:

        Uma \emph{classe} é um \emph{template} extensível para a criação de
        objetos. As classes podem ser instanciadas através de uma chamada para
        o método construtor, uma subrotina responsável pela inicialização de
        dados de objetos. As entidades declaradas como públicas podem ser
        acessadas através dessas instâncias. De acordo com Sebesta, esta
        maneira é mais limpa e direta que o encapsulamento em Ada.

        Classes possuem membros de dados, que armazenam estado e informação, e
        membros de subrotina, implementações de comportamento, também chamados
        de métodos em linguagens orientadas a objeto. Além disso, membros de
        dados podem ser categorizados em membros de classe, independentes de
        instanciação e membros de instância, que podem assumir valores
        diferentes por cada objeto. Para organizar o código e aumentar a
        legibilidade, C++ permite separar a definição de uma classe e sua
        implementação.

        A diferença entre as classes das linguagens orientadas a objetos e
        pacotes em Ada é que classes geram tipos.

        \item
        No caso dos módulos:

        Um \emph{módulo} é uma abstração que tem como como finalidade a
        granularização do código, é implementado de forma a dividir as
        interfaces e as implementações em arquivos diferente assim como na
        ocorre na linguagem C. Ao final permite uma melhor organização do
        código e compilação separada.

        Na nossa linguagem não utilizaremos módulos, pois como ele se destina a
        um público iniciante, o caráter dos programas é simples, não
        necessitando das abstrações de módulo.

\end{enumerate}


        \item
        Todas são encapsulamentos que permitem a abstração de dados e funções do
código. No caso do tipo de dados, temos a noção de uma implementação de um
grupo de funções, das quais o usuário só conhece aquelas que estão na definição
do grupo. Classes de objetos são uma implementação em cima dessa noção criada a
partir de TAD.

No caso de pacotes e módulos, ambos tem a função tanto de organização do código
e separação da compilação. A diferença se encontra na implementação de ambos.

        \item
        Em relação aos pacotes, como foi dito anteriormente, eles serão implementados
no mesmo estilo da Ada, porém com cabeçalho e corpo no mesmo arquivo. Para
chamar um pacote em um arquivo, será utilizado o comando \inlinecode{import
<nome_do_arquivo>}

Anotações
- Criação de pacotes - porém, códigos monolíticos não são muito problema


        \item
        - Tipos e pacotes mais genéricos, tipos parametrizados

        \item
        Anotações
- Quais são as estruturas que permitem que você guarde e oculte
- namespaces (encapsulamento de nomes)

Quando definimos tipos abstratos de dados, classes, módulos
e pacotes, usamos de encapsulamentos  mínimos, ou seja, coleções pequenas
de código e data logicamente relacionado, de modo a criar uma abstração 
para o usuário. Essas coleções são chamadas de encapsulamento.
Mas quando o código começa a crescer na casa das milhares de linhas de
código, utilizamos esse conceito de forma expandida. 

Em C, uma coleção de funções e data relacionadas podem ser colocadas em 
um arquivo compilado separadamente, age como uma biblioteca.A Interface
 para tal arquivo é colocada em um arquivo separado, chamado de header 
file. Tal código pode ser incluído em um programa cliente através do 
/emph{#include}.Esse tipo de encapsulamento gera algumas inseguranças,
pois o código pode ser simplesmente copiado e colado, ignorando o header,
o que funciona, mas acarreta uma perda da documentação de dependência. 

Em C++ 

    \end{enumerate}
\end{document}

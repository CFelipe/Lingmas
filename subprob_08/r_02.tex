Ainda\begin{enumerate}
        \item
        No caso de tipos abstratos de dados:

        Um \emph{tipo abstrato de dado} tem a representação do tipo dos objetos
        ocultas para as unidades de programa que utilizam esse tipo. Dessa forma as
        únicas operações possíveis nesses objetos são aquelas que estão na
        definição do objeto.

        Além disso as operações do tipo e as operações definidas nos objetos
        desse tipo, que fornecem a interface desse tipo, são contidas em uma
        única unidade sintática.

        Assim,  a interface do tipo não depende da representação dos objetos
        nem da definição de suas operações. Além disso outras unidades de
        programa podem usar variáveis do tipo definido.

        Na nossa linguagem, utilizaremos tipos abstratos de dados, presentes
        tanto no tipo float, quanto em tipos criados pelo usuário.

        % Abstrai a representação e funções de manipulação dessa representação.
        % Amadurecimento de tipo abstrato de dados -> Classes. Pois permite
        % agrupar abstrações de dados e funções
		\item
		No caso de pacotes:
        % Módulos e pacotes permitem maior organização do código e sua consequente
        % repartição

        Um \emph{pacote} é como a implementação de módulo é chamada na
        linguagem Ada. Nela, os cabeçalhos e corpos dos módulos podem ser
        separados.

        Nós utilizaremos pacotes (packages). A ideia de pacotes será
        implementada semelhante à utilização de módulos em Ada, onde cabeçalho
        e corpo podem ser separados. Porém, na nossa implementação cabeçalho e
        corpo terão que estar no mesmo arquivo.

        \item
        No caso de classes:

        Uma \emph{classe} é um \emph{template} extensível para a criação de
        objetos. As classes podem ser instanciadas através de uma chamada para
        o método construtor, uma subrotina responsável pela inicialização de
        dados de objetos. As entidades declaradas como públicas podem ser
        acessadas através dessas instâncias. De acordo com Sebesta, esta
        maneira é mais limpa e direta que o encapsulamento em Ada.

        Classes possuem membros de dados, que armazenam estado e informação, e
        membros de subrotina, implementações de comportamento, também chamados
        de métodos em linguagens orientadas a objeto. Além disso, membros de
        dados podem ser categorizados em membros de classe, independentes de
        instanciação e membros de instância, que podem assumir valores
        diferentes por cada objeto. Para organizar o código e aumentar a
        legibilidade, C++ permite separar a definição de uma classe e sua
        implementação.

        A diferença entre as classes das linguagens orientadas a objetos e
        pacotes em Ada é que classes geram tipos.

        \item
        No caso dos módulos:

        Um \emph{módulo} é uma abstração que tem como como finalidade a
        granularização do código, é implementado de forma a dividir as
        interfaces e as implementações em arquivos diferente assim como na
        ocorre na linguagem C. Ao final permite uma melhor organização do
        código e compilação separada.

        Na nossa linguagem não utilizaremos módulos, pois como ele se destina a
        um público iniciante, o caráter dos programas é simples, não
        necessitando das abstrações de módulo.
\end{enumerate}
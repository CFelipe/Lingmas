\enumerate{
    \item
        \item
        \item No caso de tipos abstratos de dados:

        Um \emph{Tipo abstrato de dado} tem a representação do tipo dos objetos
        ocultas para as unidades de programa que utilizam esse tipo. Dessa forma as
        únicas operações possíveis nesses objetos são aquelas que estão na
        definição do objeto.

        Além disso as operações do tipo e as operações definidas nos objetos
        desse tipo, que fornecem a interface desse tipo, são contidas em uma
        única unidade sintática.

        Assim,  a interface do tipo não depende da representação dos objetos
        nem da definição de suas operações. Além disso outras unidades de
        programa podem usar variáveis do tipo definido.

        Na nossa linguagem, utilizaremos tipos abstratos de dados, presentes
        tanto no tipo float, quanto em tipos criados pelo usuário.
% Abstrai a representação e funções de manipulação dessa representação.
% Amadurecimento de tipo abstrato de dados -> Classes. Pois permite agrupar
% abstrações de dados e funções 
        \item

% Módulos e pacotes permitem maior organização do código e sua consequente
% repartição 
        \item No caso de pacotes:

        Um \emph{Pacote} é como a implementação de módulo é chamada na
        linguagem Ada. Nela, os cabeçalhos e corpos dos módulos podem ser
        separados.

        Nós utilizaremos pacotes (packages). A ideia de pacotes será
        implementada semelhante à utilização de módulos em Ada, onde cabeçalho
        e corpo podem ser separados. Porém, na nossa implementação cabeçalho e
        corpo terão que estar no mesmo arquivo.
        \item
}

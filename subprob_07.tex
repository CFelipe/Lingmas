\documentclass[12pt, a4paper]{article}
\usepackage[utf8]{inputenc}
\usepackage[brazilian]{babel} % Hifenização e dicionário
\usepackage[left=3.00cm, right=2.00cm, top=3.00cm, bottom=2.00cm]{geometry}
\usepackage{enumitem} % Para itemsep etc
\usepackage{longtable} % Dependência do longtabu
\usepackage{tabu} % Para melhor criação de tabelas
\usepackage{color}
\usepackage{parskip} % Linha em branco entre parágrafos em vez de recuo
\usepackage{graphicx}
\usepackage[breaklinks]{hyperref}

\tabulinesep=0.75ex % Espaçamento interno das tabelas

\begin{document}
    \begin{center}
        \textsc{Universidade Federal do Rio Grande do Norte} \\
        \textsc{Departamento de Informática e Matemática Aplicada}
    \end{center}

    \bigskip

    \begin{tabular}{@{}ll@{}}
            \emph{Disciplina:} & DIM0437 --- Linguagens de Programação:
                                 Conceitos e Paradigmas \\
            \emph{Docente:}    & Umberto Souza da Costa \\
            \emph{Discentes:}  & Dogival Ferreira da Silva Junior \\
                               & Felipe Cortez de Sá \\
                               & Gabriel Sebastian von Conta \\
                               & Phellipe Albert Volkmer \\
                               & Vinícius Araújo Petch
    \end{tabular}

    \bigskip

    \begin{center}
        \large \textbf{Subproblema 7} --- Subprogramas
    \end{center}

    \section{Problema}
        \subsection{Produto do problema}
        Definição da sintaxe e semântica intuitiva dos mecanismos que regem as
        formas de abstração de processamento (procedimentos e funções) da
        linguagem de programação a ser definida pelo grupo. Incluir a
        representação destas abstrações, assim como os mecanismos de passagem
        de parâmetros e a implementação sugerida.

        \subsection{Questões}
        \begin{enumerate}
            \item
            Como serão definidos os procedimentos e funções de sua linguagem?
            Note que estes conceitos são diferentes, embora sejam tratados de
            forma unificada por algumas linguagens de programação. Quais são as
            diferenças entre esses conceitos?

            \item
            Como serão definidos os parâmetros da linguagem? Quais dados
            poderão ser colocados como argumentos em chamadas a procedimentos e
            funções? Nomes de subprogramas poderão ser utilizados como
            parâmetros?

            \item
            Quais serão as formas de passagem de parâmetros e como serão
            implementadas?

            \item
            Sua linguagem verificará os tipos de parâmetros dos subprogramas?

            \item
            A linguagem terá subprogramas sobrecarregados ou genéricos?

            \item
            A linguagem deverá ter compilação separada ou independente?

            \item
            Sua linguagem dará suporte a co-rotinas?
        \end{enumerate}

    \section{Resoluções}
    % TODO

\end{document}

Uma \textbf{abstração} é uma simplificação útil para o programador, uma vez que
permite representar uma entidade, filtrando as informações e operações,
aproveitando apenas o que é relevante.

Por exemplo, uma entidade \inlinecode{grupo de lpcp} possui os atributos gerais
componentes e nota e trabalhos, mas também possui características únicas a
certos grupos, como por exemplo \inlinecode{possui gordos}. Criamos então um objeto
\inlinecode{G2} que precisa do atributo específico \inlinecode{possui gordos}, mas não
\inlinecode{componentes}, \inlinecode{nota} e \inlinecode{trabalhos}, herdados da entidade
\inlinecode{grupo de lpcp}.
